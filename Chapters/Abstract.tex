\documentclass[../hodai_thesis.tex]{subfiles}
\begin{document}
\begin{center}
\LARGE \textbf{Abstract}
\end{center}

In this thesis we present an approach, a proof-of-concept tool, and experiments, for reactive scheduling of computations in software control systems. 
Such scheduling techniques are needed, for example, due to the growing use of computer vision algorithms in real-time, as sensors.
The goal of developing reactive schedulers is to combine the dynamicity and efficiency of desktop operating systems with the predictability of real-time operating systems.
The term reactive here (and in the title of the thesis) refers to the ability of the scheduler to adapt the scheduling characteristics to the physical conditions at run-time. 
Specifically, we propose mechanisms that allow the scheduler continuously reacts to the environmental inputs and to the internal system state. 

We propose an extension of the automata based scheduling approach, as implemented, e.g., in the RTComposer~\cite{RTComposer} tool, with an addition of guards on the
transitions. This addition allows for schedulers that react to data from the plants and from the controllers. Automata are reach interfaces for requirements specifications of tasks. They
allows a dynamic schedule without breaking the predictability of real-time schedulers. The scheduler reactivity that we add allows for efficient schedules that 
alocated resources only when needed, based the inputs. The saved processing time can be used for other tasks or for lowering the cost of the system.

The main contributions of this thesis relative to the earlier work of RTComposer~\cite{RTComposer} and GameComposer~\cite{Merav}, are twofolds: (1) We develop methodologies for
creating ``enviroment''-depended component, using Kalman and Complementary filters that provide valuable data that guides these automata. (2) We validate the concepts with a real-life 
case study and with computer simulations.

The case study we use in this thesis is the development of a software controller that stabilizes a drone in front of a window. We show how a vision based sensor can be used with a varying resolution, i.e.,
computation load is controlled by taking images in reduced resolution when the state of the controlled loop allows. We developed this controller using the well known 
open-source autopilot software ArduPilotMega (APM)~\cite{APM} with minimum changes of the original software. This allows us to draw some conclusions regarding the possibility of integrating the extended 
automata based approach proposed here with the state of the art control systems.

\end{document}