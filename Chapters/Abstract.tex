\documentclass[../GameComposer.tex]{subfiles}
\begin{document}
\begin{center}
\LARGE \textbf{Abstract}
\end{center}

\todo{abstract from paper}
    We present an approach to reactive scheduling of computations in software control systems. The main motivation comes form the growing use of computer vision algorithms in real-time, as sensors. The term reactive here refers to the ability of the scheduler to adapt the schedules dynamically based on physical conditions. We propose an extension of the automata based scheduling approach with an addition of guards to transitions that allow for reactive specifications. We develop a methodology for using Kalman filters to provide data that guides these automata, and demonstrate the combined approach in simulations and with a case study. The case study is a development of a software that stabilizes a drone in front of a window using a vision based sensor with a time-varying resolution, i.e., computation load is controlled by taking images in reduced resolution when the state of the controlled loop allows.

\end{document}